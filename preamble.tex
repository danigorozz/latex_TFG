\documentclass[a4paper, 12pt]{book}

% Paquetes básicos de LATEX
\usepackage{amsmath, amssymb}
\usepackage[T1]{fontenc}
\usepackage[utf8]{inputenc}
\usepackage[spanish]{babel}

% Título y autor
\title{Título del proyecto}
\author{Daniel Gómez}

% Para cambiar márgenes
\usepackage[a4paper]{geometry}
\geometry{left=3cm, right=3cm}

% Para quitar encabezado, pie de página y numeración en páginas vacías
\usepackage{emptypage}

% Para permitir el parámetro H en las figuras y tablas
\usepackage{float}

% Para añadir figuras y su path
\usepackage{graphicx}
\graphicspath{{figures/}}

% Para modificar encabezados y pie de página
\usepackage{fancyhdr}
\fancyhf{} % Limpia los encabezados y pie de página
\lhead[\leftmark]{Daniel G. R.}
\rhead[Daniel G. R.]{\rightmark}
\cfoot[\thepage]{\thepage}

% Añadir línea separatoria en el encabezado
\renewcommand{\headrulewidth}{0.4pt}

% Paquete para añadir enlaces entre referencias
\usepackage[hidelinks]{hyperref}

% Comando creado para añadir figuras más rápido y más limpio
\newcommand{\addimage}[4]{
\begin{figure}[h]
\begin{center} \label{fig:#2}
\includegraphics[width=#4\textwidth]{#1}
\end{center}
\caption{#3}
\end{figure}
}

