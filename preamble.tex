\documentclass[a4paper, 12pt]{book}

% Paquetes básicos de LATEX
\usepackage{amsmath, amssymb}
\usepackage[T1]{fontenc}
\usepackage[utf8]{inputenc}
\usepackage[spanish]{babel}

% Título y autor
\title{Título del proyecto}
\author{Daniel Gómez}

% Para cambiar márgenes
\usepackage[a4paper]{geometry}
\geometry{left=3cm, right=3cm, bottom=4cm}

% Para quitar encabezado, pie de página y numeración en páginas vacías
\usepackage{emptypage}

% Para permitir el parámetro H en las figuras y tablas
\usepackage{float}

% Para poder poner varias imágenes en la misma figura
\usepackage{subfig}

% Para añadir figuras y su path
\usepackage{graphicx}
\graphicspath{{figures/}}

% Para modificar encabezados y pie de página
\usepackage{fancyhdr}
\fancyhf{} % Limpia los encabezados y pie de página
\lhead[\leftmark]{}
\rhead[]{\rightmark}
\cfoot[\thepage]{\thepage}

% Añadir línea separatoria en el encabezado
\renewcommand{\headrulewidth}{0.4pt}

% Paquete para añadir enlaces entre referencias
\usepackage[hidelinks]{hyperref}

% Comando creado para añadir figuras más rápido y más limpio
\newcommand{\addimage}[4]{
\begin{figure}[h]
\begin{center} \label{fig:#2}
\includegraphics[width=#4\textwidth]{#1}
\end{center}
\caption{#3}
\end{figure}
}

\usepackage{color}
\usepackage{listings}
\definecolor{codegreen}{rgb}{0,0.6,0}
\definecolor{codegray}{rgb}{0.5,0.5,0.5}
\definecolor{codeyellow}{rgb}{0.6,0.6,0}
\definecolor{codepurple}{rgb}{0.58,0,0.82}
\definecolor{backcolour}{rgb}{0.97,0.97,0.95}
\definecolor{deepblue}{rgb}{0,0,0.5}
\definecolor{deepred}{rgb}{0.6,0,0}
\definecolor{deepgreen}{rgb}{0,0.5,0}

\lstdefinelanguage{Python2}{
  language     = Python,
  morekeywords = {self, True, False},
}

\lstdefinestyle{pythonstyle}{
 backgroundcolor=\color{backcolour},   
 commentstyle=\color{codegreen},
 keywordstyle=\color{deepblue},
 numberstyle=\tiny\color{codegray},
 stringstyle=\color{codepurple},
 basicstyle=\ttfamily\footnotesize,
 breakatwhitespace=false,         
 breaklines=true,                 
 captionpos=b,                    
 keepspaces=true,                 
 numbers=left,                    
 numbersep=5pt,                  
 showspaces=false,                
 showstringspaces=false,
 showtabs=false,                  
 tabsize=2,
 keywords={self},
 emph={__init__, __name__},          % Custom highlighting
 emphstyle=\ttfamily\color{deepred},    % Custom highlighting style
}

\lstdefinestyle{cppstyle}{
 backgroundcolor=\color{backcolour},   
 commentstyle=\color{codegreen},
 keywordstyle=\color{deepblue},
 numberstyle=\tiny\color{codegray},
 stringstyle=\color{codepurple},
 basicstyle=\ttfamily\footnotesize,
 breakatwhitespace=false,         
 breaklines=true,                 
 captionpos=b,                    
 keepspaces=true,                 
 numbers=left,                    
 numbersep=5pt,                  
 showspaces=false,                
 showstringspaces=false,
 showtabs=false,                  
 tabsize=2,
 keywords={self},
 emph={__init__, __name__},          % Custom highlighting
 emphstyle=\ttfamily\color{deepred},    % Custom highlighting style
}


\newcommand\pythonexternal[2][]{{
\pythonstyle
\lstinputlisting[#1]{codes/#2}}}