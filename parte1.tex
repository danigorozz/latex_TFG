\part{Introducción}

\chapter{Introducción y visión general}

\section{Motivación y justificación}

La robótica móvil ha ido evolucionando a un ritmo vertiginoso desde que comenzó a desarrollarse. En los años sesenta se diseñó el apodado como `SHAKEY'' o ``el primer robot inteligente móvil del mundo'' según el IEEE (Instituto de Ingeniería Eléctrica y Electrónica). Este robot era el primero de su generación capaz de navegar en un entorno no controlado, sirviéndose de múltiples dispositivos que le proporcionaban información sobre la distribución de los elementos que le rodeaban, dándole la posibilidad al robot de evitarlos durante el trayecto.\\

Un robot autónomo móvil (ARM) es un robot capaz de navegar a través de un entorno sin supervisión directa por parte de un operador humano ni la necesidad de disponer de un ruta fijada previamente. Para poder realizarlo, los robots disponen de infinidad de sensores que les permiten percibir e interpretar el entorno, dotando al robot de la capacidad de navegar por el mismo evitando obstáculos, tanto fijos como móviles. Para poder realizar una navegación de alto nivel que les permita ir desde un punto origen a otro punto destino, los robots se sirven de mapas de ocupación que son generados en una primera fase de inspección del entorno, empleando los mismos sensores que luego servirán para la navegación autónoma. Tradicionalmente, estos mapas de ocupación son generados empleando un láser 2D o LIDAR (dado su gran alcance y amplio campo de visión), generando por tanto mapas 2D que especifican aquellas zonas del entorno que están libres de obstáculos y por tanto susceptibles de navegación, y aquellas que no lo están. No obstante, dado que un láser 2D no puede distinguir entre los objetos detectados, los mapas de ocupación 2D generados con este sensor contienen, no sólo los elementos estructurales del entorno como paredes, puertas, columnas, etc., si no que incluyen multitud de objetos como mesas y sillas, camas, cajas e incluso personas si estaban presentes en el momento de generar el mapa.\\

\addimage{IlustracionErroresMapa.png}{errores_laser}{Errores típicos con el barrido de un láser 2D}{0.7}

En la figura X se muestra un ejemplo de un mapa generado de un entorno virtual a través de un láser 2D de 360º. Como se puede observar hay zonas, marcadas con numeración, en las que el mapa no representa fielmente la estructura del entorno. De modo que, si los objetos que no forman parte de la estructura del entorno modifican su posición después de la primera fase de análisis, el mapa de ocupación generado no podría ser utilizado por el robot para la navegación. \\

Dado que estos métodos de generación de mapas de ocupación mediante láser 2D sufren frente a la variabilidad en la distribución, surge la idea de utilizar otros sensores que permitan discernir entre los distintos elementos que pudieran formar parte del entorno. Los sensores RGB-D son cámaras que, además de proporcionar información sobre el color, ofrecen la distancia a la que se encuentra cada píxel captado en la imagen con respecto a la cámara. Por lo tanto, ofrecen información en 3D del dominio, lo que abre más posibilidades a la hora de percibir e interpretar la información captada mediante los sensores.\\

En este proyecto se pretende analizar una posible solución al problema de fiabilidad en la representación de los mapas de ocupación realizados con sensores LIDAR mediante el uso de sensores RGB-D para generar mapas de ocupación fieles a la estructura invariable del entorno.\\

\section{Objetivo}

El objetivo de este trabajo es utilizar la información proporcionada por sensores RGB-D y utilizarla para la generación floorplans, solucionando problemas de variabilidad en el entorno que perjudican seriamente la fiabilidad  del mapeado mediante sensores láser.\\

Se obtendrá información acerca de la profundidad de los píxeles captados y se escogerán aquellos puntos menos restrictivos (más alejados) para generar un láser artificial que permita la creación del mapa de ocupación mediante algoritmos de mapeado.\\

Finalmente, se hará una comparativa de los resultados obtenidos únicamente con el LIDAR con los obtenidos mediante el procesamiento de la información de la cámara.\\